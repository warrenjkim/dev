\documentclass{report}
\usepackage{amsmath}
\usepackage{enumitem}
\usepackage{amssymb}
\usepackage{listings}
\usepackage{hyperref}
\usepackage{booktabs}

\newcommand{\textib}[1]{\textit{\textbf{{#1}}}}
\newcommand{\code}[1]{\texttt{{#1}}}
\newcommand{\note}[1]{\textib{Note:} \textit{{#1}}}

\lstset{
    basicstyle=\ttfamily,              % the size of the fonts used for the code
    breaklines=true,                   % sets automatic line breaking
    tabsize=2,                         % sets default tab size to 4 spaces
    showstringspaces=false,            % instructs LaTeX not to put in spaces at visible spaces
    frame=single,                      % adds a frame around the code
}

\begin{document}
\tableofcontents
\newpage
\chapter{Preface}
These are my notes for Rust. 

\chapter{Variables}
\section{Mutability}
Variables by default are \textit{immutable} in Rust. That is, we have to specify that we want a 
certain variable to be mutable.
\begin{lstlisting}
    let var_name = value
    let mut var_name = value
\end{lstlisting}
A couple notes:

\begin{enumerate}[label=\textit{(\roman*)}]
    \item Trying to mutate immutable variables (e.g. \code{let x...}) will result in an
        \textit{immutability error} in the compiler.
    \item Adding \code{mut} to a variable name (e.g. \code{let mut x...}) indicates that
        the variable is indeed mutable.
\end{enumerate}

\section{Constants}
Constants are (by definition) \textit{immutable}, and can \textib{not} be made mutable using 
\code{mut}. They are declared as follows:
\begin{lstlisting}
    const CONST_NAME: u32 = 60 * 60 * 3;
\end{lstlisting}
Note that constants require the type \textit{and} value to be specified. Additionally, constants
can only be set to constant expressions; i.e. you \textib{cannot} set a constant to something 
computed at runtime.
\newline
\newline
\note{Naming convention is to use uppercase snake case.}


\section{Shadowing}
You can \textit{shadow} variables in Rust. Consider the following function in \code{main.rs}:

\begin{lstlisting}
1  fn main() {
2    let x = 5;
3    let x = x + 1;
4
5    {
6      let x = x * 2;
7      println!("x in the inner scope is: {x}");
8    }
9    println!("the value of x is: {x}");
10 }
\end{lstlisting}
\begin{enumerate}[label=,leftmargin=0pt]
    \item At (2), \code{x = 5}.
    \item At (3), \code{x = 5 + 1 = 6}.
    \item At (6), \code{x = 6 * 2 = 12}.
    \item At (7), we print the \textib{shadowed} \code{x} (the \code{x} in the \textib{inner} scope
        [\code{x = 12}]).
    \item At (9), we print the \code{x} as normal.
\end{enumerate}


\subsection{\texttt{mut} v. Shadowing with \texttt{let}} 
We can also shadow variables using \code{let}. The following is perfectly legal:
\begin{lstlisting}
  let spaces = "     ";
  let spaces = spaces.len();
\end{lstlisting}
Where the first and second \code{spaces} is a string and number type respectively.
\newline
\newline
However, doing the same using \code{mut} will produce a mismatched types error in the compiler:
\begin{lstlisting}
  let mut spaces = "     ";
  spaces = spaces.len();
\end{lstlisting}
This is because we are not allowed to mutate a variable's type.


\subsection{Summary}
Rust variables are \textib{immutable} by default and must be specified (using \code{mut}) if they
are to be mutated. 
\newline
\newline
Constants in Rust \textib{require} a type annotation and can only be assigned to
\textib{constant} expressions (e.g. \code{10 * 10}). 
\newline
\newline
Rust has shadowing with the expected behavior. However, a common thing to do in Rust (apparently)
is to shadow a variable via \code{let}. Note that we \textib{cannot} do this with variables 
specified with \code{mut}.





\section{Data Types}
Rust is \textib{statically typed}; i.e. we know the types of \textit{all} variables at 
\textib{compile time}. There are two data type subsets: \textib{scalar} and \textib{compound}. We
can explicitly define the type of a variable as such:
\begin{lstlisting}
let guess: u32 = "42".parse().expect("Not a number!");
\end{lstlisting}
The \texttt{: u32} explicitly defines that \texttt{guess}  is a numberic type.


\subsection{Scalar Types}
A \textib{scalar} represents a \textit{single} value. Rust has
\begin{enumerate}[label=\textit{(\roman*)}]
    \item integer
    \item floating-point
    \item booleans
    \item character
\end{enumerate}

\subsubsection{Integers}
An \textit{integer} is a number without a fractional component (i.e. $\mathbb{Z}$). We can specify
the \textit{length} as well as whether or not it is \textit{signed} or \textit{unsigned}. The table
is as follows:
\begin{center}
\centering
\begin{tabular}{ccc}
\toprule
Length & Signed & Unsigned \\
\midrule
8-bit & \texttt{i8} & \texttt{u8} \\
16-bit & \texttt{i16} & \texttt{u16} \\
32-bit & \texttt{i32} & \texttt{u32} \\
64-bit & \texttt{i64} & \texttt{u64} \\
128-bit & \texttt{i128} & \texttt{u128} \\
arch & \texttt{isize} & \texttt{usize} \\
\bottomrule
\end{tabular}
\end{center}
Note that signed numbers are stored using \textit{two's complement}. The \texttt{arch} length depends
on the \textit{architecture} of your computer.
\newline
\newline
We can write integer literals in any of the following ways:
\newline
\begin{center}
\begin{tabular}{cc}
\toprule
Length & Signed \\
\midrule
Decimal & \texttt{1\_000} \\
Hex & \texttt{0xff} \\
Octal & \texttt{0o77} \\
Binary & \texttt{0b1111\_0000} \\
Byte (\texttt{u8} only!) & \texttt{b'A'} \\
\bottomrule
\end{tabular}
\end{center}
Note here that the \texttt{\_} is just a \textit{visual} separator for readability.
















\chapter{Learning Journal}

\section{10/18/23}
\subsection*{Brief}
\begin{enumerate}[label=\textit{(\roman*)}]
    \item Started Rust notes.
    \item Currently on \texttt{3.2 Data Types}.
    \item Learned about how Rust variables worked.
    \item Got PTSD from CS131 about how shadowing works.
\end{enumerate}

\subsection*{Major Takeaways}
\begin{enumerate}[label=\textit{(\roman*)}]
    \item Rust variables are immutable by default, and must be specified that they are mutable.
    \item Constants require a type annotation.
    \item Rust has shadowing with expected behavior.
    \item Rust uses shadowing a lot via \code{let}.
    \item Variables with \code{mut} cannot be shadowed with \textit{(iv)}.
\end{enumerate}



\end{document}
